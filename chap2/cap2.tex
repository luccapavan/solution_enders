\chapter{Modelos de séries de tempo estacionárias}

\begin{enumerate}
	\item %1
 $w_t=\frac{1}{4}\varepsilon_{t}+\frac{1}{4}\varepsilon_{t-1}+\frac{1}{4}\varepsilon_{t-2}+\frac{1}{4}\varepsilon_{t-3}$
	
		\begin{enumerate}
			\item %1a
			\begin{enumerate}
				\item \begin{align*}E(w_t)&=E(\frac{1}{4}\varepsilon_{t}+\frac{1}{4}\varepsilon_{t-1}+\frac{1}{4}\varepsilon_{t-2}+\frac{1}{4}\varepsilon_{t-3})\\
				&=\frac{1}{4}E(\varepsilon_{t})+\frac{1}{4}E(\varepsilon_{t-1})+\frac{1}{4}E(\varepsilon_{t-2})+\frac{1}{4}E(\varepsilon_{t-3})=0
				\end{align*}
				
				\item 
				\begin{align*}E(w_t|\varepsilon_{t-3}=\varepsilon_{t-2}=1)&=E(\frac{1}{4}\varepsilon_{t}+\frac{1}{4}\varepsilon_{t-1}+\frac{1}{4}(1)+\frac{1}{4}(1))\\
					&=\frac{1}{4}E(\varepsilon_{t})+\frac{1}{4}E(\varepsilon_{t-1})+\frac{1}{4}+\frac{1}{4}=\frac{1}{2}
				\end{align*}
			\end{enumerate}
			\item %1b
			
			\begin{enumerate}
				\item 
				\begin{align*}
					\text{var}(w_t)&=E(w_t^2)-E(w_t)^2, \;\;\; E(w_t)=0\\
					\Rightarrow E(w_t^2)&=E([\frac{1}{4}\varepsilon_{t}+\frac{1}{4}\varepsilon_{t-1}+\frac{1}{4}\varepsilon_{t-2}+\frac{1}{4}\varepsilon_{t-3}]^2)\\
				&=\frac{1}{16}E(\varepsilon_{t}^2)+\frac{1}{16}E(\varepsilon_{t-1}^2)+\frac{1}{16}E(\varepsilon_{t-2}^2)+\frac{1}{16}E(\varepsilon_{t-3}^2), \;\; \text{pois } E(\varepsilon_{t},\varepsilon_{t-s})=0\; \forall \;s>0\\
				&=\frac{1}{16}\sigma^2+\frac{1}{16}\sigma^2+\frac{1}{16}\sigma^2+\frac{1}{16}\sigma^2=\frac{1}{4}\sigma^2
				\end{align*}		
				\item
				\begin{align*}
				\text{var}(w_t|\varepsilon_{t-3}=\varepsilon_{t-2}=1)&=E(w_t^2|\varepsilon_{t-3}=\varepsilon_{t-2}=1)-E(w_t|\varepsilon_{t-3}=\varepsilon_{t-2}=1)^2,\\
				E(w_t|\varepsilon_{t-3}=\varepsilon_{t-2}=1)&=\frac{1}{2}\\
				\Rightarrow	\text{var}(w_t|\varepsilon_{t-3}=\varepsilon_{t-2}=1)&=E(w_t^2|\varepsilon_{t-3}=\varepsilon_{t-2}=1)-\frac{1}{4}\\
				\\
				E(w_t^2|\varepsilon_{t-3}=\varepsilon_{t-2}=1)&=E([\frac{1}{4}\varepsilon_{t}+\frac{1}{4}\varepsilon_{t-1}+\frac{1}{2}]^2)\\
				&=\frac{1}{16}E(\varepsilon_t^2)+\frac{1}{16}E(\varepsilon_{t-1}^2)+\frac{1}{4}\\
				&=\frac{1}{4}+\frac{1}{8}\sigma^2\\
				\text{var}(w_t|\varepsilon_{t-3}=\varepsilon_{t-2}=1)&=\frac{1}{4}+\frac{1}{8}\sigma^2-\frac{1}{4}=\frac{1}{8}\sigma^2
				\end{align*}		
			\end{enumerate}
			
			\item %1c
				\begin{enumerate}
					\item 
					\begin{align*}
					\text{cov}(w_t,w_{t-1})&=E(w_tw_{t-1})-E(w_t)E(w_{t-1})\\
					&=E(w_tw_{t-1}), \text{pois } E(w_{t-s})=0\; \forall\; s\geq0\\
					E(w_tw_{t-1})&=E[(\frac{1}{4}\varepsilon_{t}+\frac{1}{4}\varepsilon_{t-1}+\frac{1}{4}\varepsilon_{t-2}+\frac{1}{4}\varepsilon_{t-3})(\frac{1}{4}\varepsilon_{t-1}+\frac{1}{4}\varepsilon_{t-2}+\frac{1}{4}\varepsilon_{t-3}+\frac{1}{4}\varepsilon_{t-4})]\\
					&=\frac{3}{16}\sigma^2, \;\;\;\text{pois } E(\varepsilon_{t},\varepsilon_{t-s})=0\; \forall \;s>0
					\end{align*}
					
					\item
					\begin{align*}
					\text{cov}(w_t,w_{t-2})&=E(w_tw_{t-2})-E(w_t)E(w_{t-2})\\
					&=E(w_tw_{t-2})\\
					E(w_tw_{t-2})&=E[(\frac{1}{4}\varepsilon_{t}+\frac{1}{4}\varepsilon_{t-1}+\frac{1}{4}\varepsilon_{t-2}+\frac{1}{4}\varepsilon_{t-3})(\frac{1}{4}\varepsilon_{t-2}+\frac{1}{4}\varepsilon_{t-3}+\frac{1}{4}\varepsilon_{t-4}+\frac{1}{4}\varepsilon_{t-5})]\\
					&=\frac{1}{8}\sigma^2
					\end{align*}
					
					\item 
					\begin{align*}
						\text{cov}(w_t,w_{t-5})&=E(w_tw_{t-5})-E(w_t)E(w_{t-5})\\
						&=E(w_tw_{t-5})\\
						E(w_tw_{t-5})&=E[(\frac{1}{4}\varepsilon_{t}+\frac{1}{4}\varepsilon_{t-1}+\frac{1}{4}\varepsilon_{t-2}+\frac{1}{4}\varepsilon_{t-3})(\frac{1}{4}\varepsilon_{t-5}+\frac{1}{4}\varepsilon_{t-6}+\frac{1}{4}\varepsilon_{t-7}+\frac{1}{4}\varepsilon_{t-8})]\\
						&=0, \; \text{pois só temos termos de erro \emph{ruído branco}  de períodos diferentes.}
					\end{align*}
				\end{enumerate}
		\end{enumerate}
		
		Com este primeiro exercício podemos perceber que a variância condicional é menor do que a não condicional. Intuitivamente faz sentido, já que pelo fato de termos mais informações no exercício condicionado ($\varepsilon_{t-3}=\varepsilon_{t-2}=1$), mais correta será nossa estimativa.
		
	------------------------------------	
		
		\item[2]
	$y_t=a_0+a_2y_{t-2}+\varepsilon_{t}, \;\; |a_2|<1$
	
	\begin{enumerate}
		
		\item
		
		\begin{enumerate}
			
			\item
			
			$E_{t-2}y_t=a_0+a_2y_{t-2}$
			
			\item
			
			$E_{t-1}y_t=a_0+a_2y_{t-2}$
			
			\item
			$E_ty_{t+2}=a_0+a_2y_t$
			
			\item
			
		\begin{align*}
			y_t&=a_0+a_2y_{t-2}+\varepsilon_{t}\\
			&=a_0+a_2a_0+a_2^2y_{t-4}+a_2\varepsilon_{t-2}+\varepsilon_{t}\\
			&=a_0+a_2a_0+a_2^2a_0+a_2^3y_{t-6}+a_2^2\varepsilon_{t-4}+a_2\varepsilon_{t-2}+\varepsilon_{t}\\
			&\vdots\\
			y_t&=\frac{a_0}{1-a_2}+\sum \limits_{\substack{i=0\\
					j=2i}}^{\infty}a_2^i\varepsilon_{t-j}+Aa_2^t
		\end{align*}
	Portanto a solução particular para $\{y_t\}$ fica: 
	\begin{align*}
		y_t^p&=\frac{a_0}{1-a_2}+\sum \limits_{\substack{i=0\\
				j=2i}}^{\infty}a_2^i\varepsilon_{t-j}\\
			&=\frac{a_0}{1-a_2}+\varepsilon_t+a_2\varepsilon_{t-2}+a_2^2\varepsilon_{t-4}+a_2^3\varepsilon_{t-6}+\cdots
		\end{align*}
	
	\begin{align*}
		\text{cov}(y_t,y_{t-1})&=E(y_t-Ey_t)(y_{t-1}-Ey_{t-1})\\
		\\
		Ey_t=Ey_{t-1}&=\frac{a_0}{1-a_2}\therefore\\
			\text{cov}(y_t,y_{t-1})&=E[(\varepsilon_t+a_2\varepsilon_{t-2}+a_2^2\varepsilon_{t-4}+a_2^3\varepsilon_{t-6}+\cdots)\\
			\;\;&\times(\varepsilon_{-1}+a_2\varepsilon_{t-3}+a_2^2\varepsilon_{t-5}+a_2^3\varepsilon_{t-7}+\cdots)]\\
			&=0\\
			\Rightarrow \rho_1&=0 \therefore \phi_{11}=0
	\end{align*}

	Para encontrar $\phi_{22}$ precisamos de $\rho_2$, portanto de $\gamma_0$ e $\gamma_2$.
	
	\begin{align*}
		\gamma_0=E(y_t-Ey_t)^2&=E[(\varepsilon_t+a_2\varepsilon_{t-2}+a_2^2\varepsilon_{t-4}+a_2^3\varepsilon_{t-6}+\cdots)\\
		&\;\;\times(\varepsilon_t+a_2\varepsilon_{t-2}+a_2^2\varepsilon_{t-4}+a_2^3\varepsilon_{t-6}+\cdots)]\\
		&=\frac{\sigma^2}{1-a_2^2}\\
		\\
		\gamma_2=E(y_t-Ey_t)(y_{t-2}-Ey_{t-2})&=E[(+\varepsilon_t+a_2\varepsilon_{t-2}+a_2^2\varepsilon_{t-4}+a_2^3\varepsilon_{t-6}+\cdots)\\
		&\;\;\times(+\varepsilon_{t-2}+a_2\varepsilon_{t-4}+a_2^2\varepsilon_{t-6}+a_2^3\varepsilon_{t-8}+\cdots)]\\
		&=\frac{a_2\sigma^2}{1-a_2^2}\therefore\\
		\rho_2&=\frac{\gamma_2}{\gamma_0}=a_2\\
		\\
		\phi_{22}&=\frac{\rho_2-\rho_1^2}{1-\rho_1^2}=a_2
	\end{align*}

		\end{enumerate}
	
	\item %2b
	
	$$y_t=\frac{a_0}{1-a_2}+\sum \limits_{\substack{i=0\\	j=2i}}^{\infty}a_2^i\varepsilon_{t-j}+Aa_2^t$$
		
		Função impulso resposta: 			
		\begin{align*}
			\frac{\partial y_t}{\partial\varepsilon_{t-1}}&=0\\
			\frac{\partial y_t}{\partial\varepsilon_{t-2}}&=a_2\\
			\frac{\partial y_t}{\partial\varepsilon_{t-3}}&=0\\
			\frac{\partial y_t}{\partial\varepsilon_{t-4}}&=a_2^2\\
			&\vdots\\
			\frac{\partial y_t}{\partial\varepsilon_{t-j}}&=a_2^{i},\;\;\; i=1,2,...,\;\;j=2i.
		\end{align*}
	
Efeito de $\varepsilon_t$ sobre $\{y_t\}$:
\begin{align*}
\frac{\partial y_t}{\partial\varepsilon_{t}}&=1\\
&\vdots\\
\frac{\partial y_{t+j}}{\partial\varepsilon_{t}}&=a_2^{i},\;\;\; i=1,2,...,\;\;j=2i.
\end{align*}
	
	
	\end{enumerate}
		
		
	
	\end{enumerate}