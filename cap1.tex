\chapter{Equações em diferença}

\begin{enumerate}
	\item	
	\begin{enumerate}
		\item 
	 
	
	Encontrando a solução de Jill:
	\begin{align*}
		y_t&=a_0+a_1y_{t-1}\\
		&= a_0+a_1(a_0+a_1y_{t-2})\\
		&=a_0+a_0a_1+a_0a_1^2+\cdots+a_0a_1^{t-1}+a_1^ty_0\\
		&= a_0\sum_{i=0}^{t-1}a_1^{i}+a_1^ty_0.
	\end{align*}
	O primeiro termo da soma é uma progressão geométrica finita, então
	
	\begin{equation*}
		a_0\sum_{i=0}^{t-1}a_1^i=a_0\frac{(1-a_1^{t})}{(1-a_1)}=\frac{a_0}{(1-a_1)}-\frac{a_0a_1^{t}}{(1-a_1)}.
	\end{equation*}
	 Substituindo a P.G. finita na solução de Jill, temos
	
	\begin{align*}
		y_t&=\frac{a_0}{(1-a_1)}-\frac{a_0a_1^{t}}{(1-a_1)}+a_1^ty_0\\
		&=\frac{a_0}{(1-a_1)}-a_1^t\bigg[y_0-\frac{a_0}{(1-a_1)}\bigg].
	\end{align*}
	
	Solução para Bill:
		
	A função complementar é dada pela solução homogênea de $y_t=a_0+a_1y_{t-1}$.
	
	\begin{align*}
		y_t-a_1y_{t-1}&=0\\
		\text{se}\;\;y_t=Ab^t,\;\;y_{t-1}&=a_1Ab^{t-1}\\
		\text{então}\;\;Ab^t-a_1Ab^{t-1}&=0\\
		Ab^t({1-a_1b^{-1}})&=0 \Rightarrow b=a_1\\
	\end{align*}
	Com isso a função complementar se torna $y_t^c=Aa_1^t$, 
	
	A solução particular pode ser $y_t^p=k$, então
	
	\begin{align*}
	k&=a_0+a_1k\\
	k&=\frac{a_0}{1-a_1}\Rightarrow y^p_t=\frac{a_0}{1-a_1}
	\end{align*}
	
	então a solução geral é
	
	\begin{equation*}
		y_t=y_t^p+y_t^c=\frac{a_0}{1-a_1}+Aa_1^t
	\end{equation*}
	
	Para encontrarmos a solução definida usamos a condição inicial $y_0$ em $t=0$ na solução geral.
	
	\begin{align*}
		y_0&=\frac{a_0}{1-a_1}+Aa_1^0\\
		&= \frac{a_0}{1-a_1}+A \Rightarrow A=y_0-\frac{a_0}{1-a_1}
	\end{align*}
	Substituindo a constante arbitrária $A$ na solução geral, 
	\begin{align*}
		y_t=\frac{a_0}{1-a_1}+a_1^t\bigg[y_0-\frac{a_0}{1-a_1}\bigg],
	\end{align*}
	
	assim temos a igualdade entre as soluções de Jill e Bill.
	
\item Para Jill, se $a_1=1$, temos
	
	\begin{align*}
	y_t&=a_0+y_{t-1}\\
	&=a_0+a_0+y_{t-2}\\
	&=a_0+a_0+\cdots+a_0+y_0\\
	&=a_0t+y_0
	\end{align*}
	
	Método de Bill:
	
	\begin{align*}
		y_t^c=Ab^t \Rightarrow Ab^t&=Ab^{t-1}\\
		b&=1\\
		\Rightarrow y_t^c=A	&
	\end{align*}
	
	\begin{align*}
		y_t^p=kt \Rightarrow kt&=a_0+k(t-1)\\
		\Rightarrow a_0&=k\\
		\Rightarrow y_t^p=k+k(t-1)&=kt
	\end{align*}
	
	\begin{align*}
		y_t&=y_t^c+y_t^p=A+kt \;\text{(solução geral)}\\
		\\
	\text{em}\;\; t=0,\; y_t&=y_0 \Rightarrow y_0=A,\; \text{então}\\
	y_t&=y_0+kt\\
	\\
	\text{para}\; t=1,\; y_1&=a_0+y_0\Rightarrow y_0=y_1-a_0\\
	\text{substituindo em}\; y_t&=y_0+kt\; \text{avaliada em} \;t=1\\
	y_1&=y_1-a_0+k \Rightarrow k=a_0\\
	y_t&=a_0t+y_0 \;\text{(solução definida)}
	\end{align*}
\end{enumerate}
	------------------------------------
	
	\item \begin{enumerate}\item solução homogênea
	
	\begin{align*}
		p^*_t-(1-\alpha)p^*_{t-1}&=0\\
		 p^*_t&=Ab^t\Rightarrow Ab^t=(1-\alpha)Ab^{t-1}\\
		\Rightarrow b&=(1-\alpha)\; \;\text{e}\;\; p_t^*=A(1-\alpha)^t
	\end{align*}
	
	\item solução particular
	
	 \begin{align*}
		p_t^*&=\alpha Lp_t +(1-\alpha)Lp_t^*\\
		p_t^*&=\frac{\alpha Lp_t}{1-(1-\alpha)L}\equiv \alpha p_t \sum \limits_{i=0}^{\infty}(1-\alpha)^iL^{i+1}
		\end{align*}
		
		Somando a solução homogênea e particular
		
		\begin{align*}
			p^*_t=A(1-\alpha)^t+ \alpha p_t \sum \limits_{i=0}^{\infty}(1-\alpha)^iL^{i+1}
		\end{align*}
		
		No período $t=0$, $p_t^*=p_0^*$, então
		
		\begin{align*}
			p_0^*=A+\alpha p_0 \sum \limits_{i=0}^{\infty}(1-\alpha)^iL^{i+1}\\
			A=p_0^*-\alpha p_0\sum \limits_{i=0}^{\infty}(1-\alpha)^iL^{i+1}
		\end{align*}
		
		No período inicial $p_0^*=p_0$, então a solução definida fica
		
		\begin{align*}
			p^*_t&=\Bigg[p_0-\alpha p_0\sum \limits_{i=0}^{\infty}(1-\alpha)^iL^{i+1}\Bigg](1-\alpha)^t+ \alpha p_t \sum \limits_{i=0}^{\infty}(1-\alpha)^iL^{i+1}\\
			p^*_t&=p_0(1-\alpha)^t+ \alpha p_t \sum \limits_{i=0}^{t-1}(1-\alpha)^iL^{i+1}\\
		\end{align*}
	
	Substituindo na equação em diferença original:
	
	Antes temos,
	
\begin{align*}
		p^*_{t-1}&=p_0(1-\alpha)^{t-1}+ \alpha p_{t} \sum \limits_{i=0}^{t-1}(1-\alpha)^{i}L^{i+2}
	\end{align*}
	Então
	\begin{align*}
		p_0(1-\alpha)^t+ \alpha p_t \sum \limits_{i=0}^{t-1}(1-\alpha)^iL^{i+1}&=\alpha p_{t-1}+(1-\alpha)\bigg[p_0(1-\alpha)^{t-1}+ \alpha p_{t} \sum \limits_{i=0}^{t-1}(1-\alpha)^{i}L^{i+2}\bigg]\\
		&=\alpha p_{t-1}+(1-\alpha)p_0(1-\alpha)^{t-1}+\alpha p_t \sum \limits_{i=0}^{t-1}(1-\alpha)^{i+1}L^{i+2}\\
		\text{tratando apenas o primeiro e }&\text{o terceiro termo da expressão acima}\\
		\Longleftrightarrow \alpha p_{t-1}+\alpha p_t \sum \limits_{i=0}^{t-1}(1-\alpha)^{i+1}L^{i+2}&=\alpha p_tL(1-\alpha)^0+\alpha p_t \sum \limits_{i=0}^{t-1}(1-\alpha)^{i+1}L^{i+2}\\
		&\equiv \alpha p_t \sum \limits_{i=0}^{t-1}(1-\alpha)^{i}L^{i+1}\\
		\Rightarrow p_0(1-\alpha)^t+ \alpha p_t \sum \limits_{i=0}^{t-1}(1-\alpha)^iL^{i+1}&=	p_0(1-\alpha)^t+ \alpha p_t \sum \limits_{i=0}^{t-1}(1-\alpha)^iL^{i+1}
	\end{align*}
	Com isso os dois lados se tornam iguais, o que prova a veracidade da solução.
\end{enumerate}
	------------------------------------
	
	\item \begin{enumerate}\item 
	
	\begin{align*}
		m_t&=m+\rho m_{t-1}+\varepsilon_t\\
		\Rightarrow m_{t+1}&=m+\rho m_{t}+\varepsilon_{t+1}\\
		 m_{t+2}&=m+\rho m_{t+1}+\varepsilon_{t+2}\\
		 &\vdots\\
		  m_{t+n}&=m+\rho m_{t+n-1}+\varepsilon_{t+n}
	\end{align*}
	substituindo recursivamente o termo defasado
	\begin{align*}
	m_{t+1}	&=m+\rho m_t+\varepsilon_{t+1}\\
	m_{t+2}	&=m+\rho(m+\rho m_t+\varepsilon_{t+1})+\varepsilon_{t+2}=m+\rho m+\rho^2m_t+\rho \varepsilon_{t+1}+\varepsilon_{t+2}\\
	m_{t+3}	&=m+\rho(m+\rho m+\rho^2m_t+\rho \varepsilon_{t+1}+\varepsilon_{t+2})+\varepsilon_{t+3}\\
	&\equiv m+\rho m+\rho^2m+\rho^3m_t+\rho^2\varepsilon_{t+1}+\rho\varepsilon_{t+2}+\varepsilon_{t+3}\\
		&\vdots\\
	m_{t+n}	&=\sum\limits_{j=0}^{n-1}\rho^jm+\rho^nm_t+\sum\limits_{i=0}^{n}\rho^{n-i}\varepsilon_{t+i}
	\end{align*}
	
	\item 
	\begin{align*}
		E\bigg[m_{t+n}\bigg]&=E\bigg[\sum\limits_{j=0}^{n-1}\rho^jm+\rho^nm_t+\sum\limits_{i=0}^{n}\rho^{n-i}\varepsilon_{t+i}\bigg]\\
		&=\sum\limits_{j=0}^{n-1}\rho^jm+\rho^nm_t+E\bigg[\sum\limits_{i=0}^{n}\rho^{n-1}\varepsilon_{t+i}\bigg]\\
		&=\sum\limits_{j=0}^{n-1}\rho^jm+\rho^nm_t+\sum\limits_{i=0}^{n}\rho^{n-i}E\bigg[\varepsilon_{t+i}\bigg]\\
		E\bigg[m_{t+n}\bigg]&=\sum\limits_{j=0}^{n-1}\rho^jm+\rho^nm_t=\frac{1-\rho^{n-1}}{(1-\rho)}m+\rho^nm_t
	\end{align*}
	
	Como $m_{t+n}$ depende somente de uma variável conhecida $m_t$ e uma sequência de termos de erro $\{\varepsilon_1,\varepsilon_2,...\varepsilon_{t+n}\}$ de média zero, um modelo univariado pode ser útil para prever a oferta monetária $n$ períodos no futuro. Isto é possível estimando $\rho$ por meio de técnicas univariadas de séries temporais.
\end{enumerate}
	------------------------------------
	
	\item 
	\begin{enumerate}
		\item 
	
	 
\begin{enumerate}
	\item 
	
	
	\begin{align*}
		y_t-1.5y_{t-1}+0.5y_{t-2}&=0\\
		y_t=Ab^t\Rightarrow y_{t-1}=Ab^{t-1}\text{ e }y_{t-2}&=Ab^{t-2}\\
		\\
		Ab^t-1.5Ab^{t-1}+0.5Ab^{t-2}=0\\
		b^2-1.5b+0.5=0\\
		\\
		b_{1,2}=\frac{-a_1\pm \sqrt{a_1^2-4a_2}}{2},& \;\;\;a_1=-1.5,\;\;a_2=0.5\\
		\Rightarrow b_1=\frac{1.5+\sqrt{(-1.5)^2-4 \times 0.5}}{2}&=1\\
		b_2=\frac{1.5-\sqrt{(-1.5)^2-4 \times 0.5}}{2}&=0.5
	\end{align*}
	
	A solução homogênea fica $A_1+A_2(0.5)^t$.
	
	\item 
	
	\begin{align*}
		y_t-y_{t-2}&=0\\
	y_t=Ab^t\Rightarrow y_{t-2}&=Ab^{t-2}\\
	\\
	Ab^t-Ab^{t-2}=0, \Rightarrow b^2-1&=0,\\
	b_1=1\text{ ou }b_2=-1\\
	\end{align*}
	
	\item 
	
	\begin{align*}
	y_t-2y_{t-1}+y_{t-2}&=0\\
	\Rightarrow Ab^t-2Ab^{t-1}+Ab^{t-2}&=0\\
	b^2-2b+1&=0\\
	b_1=b_2&=1
	\end{align*}
	
		como são raízes repetidas, a solução homogênea fica $A_1+A_2t$.
	
	\item
	
	
	\begin{align*}
		y_t-y_{t-1}-0.25y_{t-2}+0.25y_{t-3}&=0\\
		\Rightarrow Ab^t-Ab^{t-1}-0.25Ab^{t-2}+0.25Ab^{t-3}&=0\\
		[b^3-b^2-0.25b+0.25&=0]\times 4\\
		(2b)^2b-(2b)^2-1b+1&=0\\
		\equiv (b-1)(2b+1)(2b-1)&=0\\
		\Rightarrow b_1=1,\;\;b_2=0.5\;\;b_3=-0.5
	\end{align*}
	
		A solução homogênea fica $A_1+A_2(0.5)^t+A_3(-0.5)^t$.
	\end{enumerate}
	\item
	
	\begin{enumerate}
	\item
	\begin{align*}
		y_t&=1.5y_{t-1}-0.5y_{t-2}+\varepsilon_t\\
		\Rightarrow y_t&=1.5Ly_t-0.5L^2y_{t}+\varepsilon_t\\
		y_t-1.5Ly_t+0.5L^2y_{t}&=\varepsilon_t\\
		(1-L)(1-0.5L)y_t&=\varepsilon_t\\
		y_t&=\frac{\varepsilon_t}{(1-L)(1-0.5L)}
	\end{align*}
	Embora a expressão $\varepsilon_t/(1-0.5L)$ seja convergente, a expressão $\varepsilon_t/(1-L)$ não é, portanto a solução retrospectiva é não convergente.
	
	\item
	 \begin{align*}
		y_t&=y_{t-2}+\varepsilon_t\\
		\Rightarrow y_t-L^2y_t&=\varepsilon_t\\
		(1-L^2)y_t&=\varepsilon_t\\
		(1-L)(1+L)y_t&=\varepsilon_t\\
		y_t&=\frac{\varepsilon_t}{(1-L)(1+L)y_t}
	\end{align*}
	A expressão $\varepsilon_t/(1-L)$ não converge, portanto a solução retrospectiva é não convergente.
	
	\item
	\begin{align*}
		y_t&=2y_{t-1}-y_{t-2}+\varepsilon_t\\
		\Rightarrow y_t-2Ly_t+L^2y_t&=\varepsilon_t\\
		(1-2L+L^2)y_t&=\varepsilon_t\\
		(1-L)(1-L)y_t&=\varepsilon_t\\
		y_t&=\frac{\varepsilon_t}{(1-L)(1-L)}
	\end{align*}
	
	Portanto a solução não converge.
	
	\item
	\begin{align*}
		y_t&=y_{t-1}+0.25y_{t-2}-0.25y_{t-3}+\varepsilon_t\\
		y_t-Ly_t-0.25L^2y_t+0.25L^3y_t&=\varepsilon_t\\
		(1-L-0.25L^2+0.25L^3)y_t&=\varepsilon_t\\
		(1-L)(1+0.5L)(1-0.5L)y_t&=\varepsilon_t\\
		y_t&=\frac{\varepsilon_t}{(1-L)(1+0.5L)(1-0.5L)}
	\end{align*}
	
	que não converge devido à expressão $\varepsilon_t/(1-L)$.	
\end{enumerate}	
\item
\begin{align*}
	y_t&=1.5y_{t-1}-0.5y_{t-2}+\varepsilon_t\\
	y_t-y_{t-1}&=1.5y_{t-1}-y_{t-1}-0.5y_{t-2}+\varepsilon_t\\
	y_t-y_{t-1}&=0.5y_{t-1}-0.5y_{t-2}+\varepsilon_t\\
	\Delta y_t&=0.5\Delta y_{t-1}+\varepsilon_t\\
	\\
	\Delta y_t&=0.5L\Delta y_{t}+\varepsilon_t\\
	\Delta y_t-0.5L\Delta y_{t}&=\varepsilon_t\\
	(1-0.5L)\Delta y_t&=\varepsilon_t\\
	\Delta y_t&=\frac{\varepsilon_t}{(1-0.5L)}\equiv \sum\limits_{i=0}^{\infty}(0.5)^i\varepsilon_{t-i}
\end{align*}

\item %d

\begin{enumerate}
	\item[ii.] 
	\begin{align*}
		y_t&=y_{t-2}+\varepsilon_t\\
		y_t-y_{t-1}&=-y_{t-1}+y_{t-2}+\varepsilon_t\\
		\Delta y_t&=-\Delta  y_{t-1}+\varepsilon_t\\
		\\
		\Delta y_t&=-\Delta  Ly_{t}+\varepsilon_t\\
		(1+L)\Delta y_t&=\varepsilon_t\\
		\Delta y_t&=\frac{\varepsilon_t}{(1+L)}\equiv \sum\limits_{i=0}^{\infty}(-1)^i\varepsilon_{t-i}\\
		\text{não converge}&\text{, portanto não há solução.}
	\end{align*}
	
	\item[iii.]
	\begin{align*}
		y_t&=2y_{t-1}-y_{t-2}+\varepsilon_t\\
		y_t-y_{t-1}&=2y_{t-1}-y_{t-1}-y_{t-2}+\varepsilon_t\\
		\Delta y_t&=\Delta y_{t-1}+\varepsilon_t\\
		\\
		\Delta y_t&=\Delta Ly_{t}+\varepsilon_t\\
		(1-L)\Delta y_t&=\varepsilon_t\\
		\Delta y_t&=\frac{\varepsilon_t}{(1-L)}\equiv\sum\limits_{i=0}^{\infty}(1)^i\varepsilon_{t-i}\\
		\text{não converge}&\text{, portanto não há solução.}
	\end{align*}
	
	\item[iv.]
	
	\begin{align*}
		y_t&=y_{t-1}+0.25y_{t-2}-0.25y_{t-3}+\varepsilon_t\\
		y_t-y_{t-1}&=0.25y_{t-2}-0.25y_{t-3}+\varepsilon_t\\
		\Delta y_t&=0.25\Delta y_{t-2}+\varepsilon_t\\
		\\
		\Delta y_t&=0.25\Delta L^2y_{t}+\varepsilon_t\\
		(1-0.25L^2)\Delta y_t&=\varepsilon_t\\
		\Delta y_t&=\frac{\varepsilon_t}{(1+0.5L)(1-0.5L)}=\frac{\varepsilon_t}{(1+0.5L)}\frac{1}{(1-0.5L)}\\
		&\equiv\Bigg[\sum\limits_{i=0}^{\infty}(-0.5)^i\varepsilon_{t-i}\Bigg]\Bigg[\sum\limits_{i=0}^{\infty}(0.5)^i(1)\Bigg]=2\sum\limits_{i=0}^{\infty}(-0.5)^i\varepsilon_{t-i}\\
		\text{que converge}&\text{, portanto esta solução existe.}
	\end{align*}	
\end{enumerate}

	\item %e 
	Já foi realizado anteriormente.
	
	\item %f
	
	\begin{align*}
		y_t&=a_0-y_{t-1}+\varepsilon_t\\
		\\
		y_0&=y_0\\
		y_1&=a_0-y_0+\varepsilon_1\\
		y_2&=a_0-y_1+\varepsilon_2=a_0-(a_0-y_0+\varepsilon_1)+\varepsilon_2=y_0+\varepsilon_2-\varepsilon_1\\
		y_3&=a_0-y_{2}+\varepsilon_3=a_0-(y_0+\varepsilon_2-\varepsilon_1)+\varepsilon_3=a_0-y_0+\varepsilon_3-\varepsilon_2+\varepsilon_1\\
		\Rightarrow y_t&=\frac{a_0+\varepsilon_t}{1+L}\equiv \sum\limits_{i=0}^{t-1}(-1)^ia_0+\sum\limits_{i=0}^{t-1}(-1)^i\varepsilon_{t-i}\\
		\text{não converge.}&
	\end{align*}		
\end{enumerate}

	------------------------------------

\item %5
\begin{enumerate}
	\item %a
	\begin{enumerate}
		\item %i
		\begin{align*} 
		y_t&=0.75y_{t-1}-0.125y_{t-2}\\
		\Rightarrow b^2-0.75b+0.125&=0\\
		b_{1,2}=\frac{-a_1\pm\sqrt{d}}{2}, \;\;d&=(a_1)^2-4a_2,\,\,a_1=-0.75, \;\;a_2=0.125\\
		d&=(-0.75)^2-4\times0.125=0.0625\\
		b_1&=\frac{0.75+\sqrt{0.0625}}{2}=0.5\\
		b_2&=\frac{0.75-\sqrt{0.0625}}{2}=0.25\\
		\\
		y_t&=A_10.5^t+A_20.25^t
	\end{align*}
	
	\item %ii
	\begin{align*}
		y_t&=1.5y_{t-1}-0.75y_{t-2}\\
		\Rightarrow b^2-1.5b+0.75&=0\\
		b_{1,2}&=\frac{1.5\pm\sqrt{(-1.5)^2-4\times 0.75}}{2}\\
		b_{1}&=\frac{1.5+\sqrt{-0.75}}{2}=0.75+i\sqrt{\frac{0.75}{4}}=0.75+i\sqrt{0.1875}\\
		b_{2}&=\frac{1.5-\sqrt{-0.75}}{2}=0.75-i\sqrt{0.1875}\\
		\\
		y_t&=A_1(0.75+i\sqrt{0.1875})^t+A_2(0.75-i\sqrt{0.1875})^t
	\end{align*}		
		
	\item %iii
	\begin{align*}
		y_t&=1.8y_{t-1}-0.81y_{t-2}\\
		\Rightarrow b^2-1.8b+0.81&=0\\
		b_{1,2}&=\frac{1.8\pm\sqrt{(-1.8)^2-4\times 0.81}}{2}\\
		d&=0\\
		b_1&=b_2=0.9\\
		\\
		y_t&=A_1(0.9)^t+A_2t(0.9)^t
	\end{align*}
	
	\item %iv
	\begin{align*}
		y_t&=1.5y_{t-1}-0.5625y_{t-2}\\
		\Rightarrow b^2-1.5b+0.5625&=0\\
		b_{1,2}&=\frac{1.5\pm \sqrt{(-1.5)^2-4\times 0.5625}}{2}\\
		d&=0\\
		b_1&=b_2=0.75\\
		\\
		y_t&=A_1(0.75)^t+A_2t(0.75)^t
	\end{align*}
\end{enumerate}
	
	\item %b
	\begin{enumerate}
		\item %5bi
	
	\begin{align*}
		y_t&=A_10.5^t+A_20.25^t,\;\;y_1=y_2=10\\
		\Rightarrow y_1=10&=A_10.5^1+A_20.25^1\\
		A_1&=20-A_20.5\\
		\\
		y_2=10&=(20-A_20.5)(0.5)^2+A_2(0.25)^2\\
		10&=(20-A_20.5)(0.25)+A_2(0.0.625)\\
		10&=(5-A_20.125)+A_2(0.0.625)\\
		5&=-A_20.0625\\
		A_2&=-80\\
		\\
		A_1&=20-(-80)0.5\\
		A_1&=60\\
		\\
		y_t&=60(0.5)^t-80(0.25)^t
	\end{align*}
	
	\begin{figure}[h]
		\centering
		\includegraphics{C:/Users/Lucca/GoogleDrive/DOUTORADO_UFPR/Macroeconometria/solution_enders/5bi.eps}
	\end{figure}
	
	\item %5bii
	
	\begin{align*}
		y_t&=A_1(0.75+i\sqrt{0.1875})^t+A_2(0.75-i\sqrt{0.1875})^t\\
		\\
		(h\pm iv)^t&=R^t(\text{cos}(\theta t)\pm i\,\text{sen}(\theta t))\\
		 \;\;\;R=\sqrt{0.75}, \;\; \text{cos}(\theta)&=\frac{1.5}{2\sqrt{0.75}}=\sqrt{0.75},\;\; \text{sen}(\theta)=\sqrt{1-\frac{(-1.5)^2}{4(0.75)}}=0.5\rightarrow \theta=\frac{\pi}{6}\\
		 \\
		\Rightarrow  y_t&=\sqrt{0.75}^t[A_1\{\text{cos}(\theta t)+ i\,\text{sen}(\theta t)\}+A_2\{\text{cos}(\theta t)- i\,\text{sen}(\theta t)\}]\\
		&=\sqrt{0.75}^t\{[A_1+A_2]\text{cos}(\theta t)+[A_1-A_2]i\,\text{sen}(\theta t)\}\\
		y_t&=\sqrt{0.75}^t\{A_5\,\text{cos}(\frac{\pi}{6} t)+\,A_6\text{sen}(\frac{\pi}{6} t)\}\text{ (forma polar) }\\
		\end{align*}
		
		\begin{align*}
		y_1=y_2&=10 \Rightarrow\\
		10&=\sqrt{0.75}\{A_5\,\text{cos}(\frac{\pi}{6} )+A_6\,\text{sen}(\frac{\pi}{6} )\}\\
			\frac{10}{\sqrt{0.75}}&=A_5\sqrt{0.75}+A_60.5\\
			A_6&=\frac{20}{\sqrt{0.75}}-A_52\sqrt{0.75}\\
			\\
			10&=\sqrt{0.75}^2\{A_5\,\text{cos}(2\frac{\pi}{6} )+A_6\,\text{sen}(2\frac{\pi}{6} )\}\\
			10&=0.75\{A_5\,\text{cos}(\frac{\pi}{3} )+(\frac{20}{\sqrt{0.75}}-A_52\sqrt{0.75})\,\text{sen}(\frac{\pi}{3} )\}\\
			10&=0.75\{A_5\,0.5+(\frac{20}{\sqrt{0.75}}-A_52\sqrt{0.75})\,\sqrt{0.75}\}\\
			15&=A_50.5+20-A_51.5\Rightarrow A_5=5 \text{ e } A_6=\frac{2\sqrt{3}}{3}\\
			\\
			y_t&=\sqrt{0.75}^t\{5\,\text{cos}(\frac{\pi}{6} t)+\,\frac{2\sqrt{3}}{3}\text{sen}(\frac{\pi}{6} t)\}
		\end{align*}
	
	\begin{figure}[h]
		\centering
		\includegraphics{C:/Users/Lucca/GoogleDrive/DOUTORADO_UFPR/Macroeconometria/solution_enders/5bii.eps}
	\end{figure}
	
	\item %5biii
	
	\begin{align*}
		y_t&=A_1(0.9)^t+A_2t(0.9)^t\\
	y_1	\Rightarrow 10&=A_1(0.9)+A_2(0.9)\\
		A_1&=\frac{10}{0.9}-A_2\\
	y_2 \Rightarrow 10&=A_1(0.9)^2+A_22(0.9)^2\\
		10&=(\frac{10}{0.9}-A_2)^2+A_21.62=\frac{100}{0.81}-2A_2\frac{10}{0.9}+A_21.62\\
		10-\frac{100}{0.81}&=-A_2\frac{20}{0.9}+A_21.62\\
		-113.45679&=-20,60222A_2\\
		\Rightarrow &A_2\approx 5.5 \text{ e } A_1\approx5.6\\
		\\
		y_t&=5.6(0.9)^t+5.5t(0.9)^t
	\end{align*}
	
	\begin{figure}[h]
		\centering
		\includegraphics{C:/Users/Lucca/GoogleDrive/DOUTORADO_UFPR/Macroeconometria/solution_enders/5biii.eps}
	\end{figure}
	
	\item %5biv
	
	\begin{align*}
		y_t&=A_1(0.75)^t+A_2t(0.75)^t\\
		y_1\Rightarrow 10&=A_1(0.75)+A_2(0.75)\\
		A_1&=\frac{40}{3}-A_2\\
		\\
		y_t&=(\frac{40}{3}-A_2)(0.75)^t+A_2t(0.75)^t\\
		y_2 \Rightarrow 10&=(\frac{40}{3}-A_2)(0.75)^2+A_22(0.75)^2\\
		10&=\frac{15}{2}-A_2\frac{9}{16}+A_2\frac{18}{16}\\
		A_2&=\frac{5}{2}\frac{16}{9}=\frac{40}{9} \text{ e } A_1=\frac{80}{9}\\
		\\
		y_t&=\frac{80}{9}(0.75)^t+\frac{40}{9}t(0.75)^t
	\end{align*}
	
	\begin{figure}[h]
		\centering
		\includegraphics{C:/Users/Lucca/GoogleDrive/DOUTORADO_UFPR/Macroeconometria/solution_enders/5biv.eps}
	\end{figure}
	
\end{enumerate}
	
\end{enumerate}

\item %6

\begin{enumerate}
	\item %6a
	
	\begin{align*}
		 y_t &= 1 + 0.7y_{t-1} - 0.1y_{t-2} + \varepsilon_t\\
		y_t -0.7y_{t_1} + 0.1y_{t-2} =0 \Rightarrow b_{1,2}&=\frac{0.7\pm \sqrt{0.7^2-4(0.1)}}{2}\\
		 b_1&=0.5\\
		 b_2&=0.2\\
		 y_t^c&=A_1^t0.5+A_20.2^t\\
		 \\
		\text{A solução particular teste neste caso é}\\
		 y_t^p&=b_0+b_1t+b_2t^2+\sum\limits_{i=0}^{\infty}\alpha_i\varepsilon_{t-i}\\
		 \Rightarrow b_0+b_1t+b_2t^2+\sum\limits_{i=0}^{\infty}\alpha_i\varepsilon_{t-i}&=1+0.7[b_0+b_1(t-1)+b_2(t-1)^2+\sum\limits_{i=0}^{\infty}\alpha_i\varepsilon_{t-1-i}]\\
		 &\;\;\;-0.1[b_0+b_1(t-2)+b_2(t-2)^2+\sum\limits_{i=0}^{\infty}\alpha_i\varepsilon_{t-2-i}]+\varepsilon_t\\
		 \text{Tratando primeiramente da parte}&\text{ estocástica da solução particular}\\
		 \Rightarrow \alpha_0&=1\\
				 \alpha_1&=0.7\alpha_0\\
				 \alpha_2&=0.7\alpha_1-0.1\alpha_0\\
				 \alpha_3&=0.7\alpha_2-0.1\alpha_1\\
				 \alpha_4&=0.7\alpha_3-0.1\alpha_2\\
				 &\vdots\\
				 \alpha_i&=0.7\alpha_{i-1}-0.1\alpha_{i-2}\\
				 \\
				 \Rightarrow  \alpha_i-0.7\alpha_{i-1}+0.1\alpha_{i-2}&=0\\
				 \alpha_i=A_1(0.2)^i&+A_2(0.5)^i\\
				 \alpha_0=1 \Rightarrow A_1&=1-A_2\\
				 \\
				  \alpha_i=(1-A_2)(0.2)^i&+A_2(0.5)^i\\
				 \alpha_1=0.7\Rightarrow 0.7&=(1-A_2)(0.2)+A_2(0.5)\\
				 0.7&=0.2-A_2(0.2)+A_2(0.5)\\
				 0.5&=A_2(0.3)\Rightarrow A_2=\frac{5}{3} \text{ e } A_1=-\frac{2}{3}\\
				 \\
				 \alpha_i&=-\frac{2}{3}(0.2)^i+\frac{5}{3}(0.5)^i
	\end{align*}
	
	Como neste problema não existe raiz unitária, não existe tendência na solução particular, então $b_1=b_2=0$. Se $y_t^p=b_0$ e tratando $\varepsilon_{t-i}=0$, podemos encontrar a parte determinística da solução particular.
	\begin{align*}
		b_0&=1+0.7b_0-0.1b_0\\
		b_0&=\frac{1}{(1-0.7+0.1)}=\frac{5}{2}\\
		\\
		\Rightarrow y_t^p&=\frac{5}{2}+\sum\limits_{i=0}^{\infty}\Bigg[\frac{5}{3}(0.5)^i-\frac{2}{3}(0.2)^i\Bigg]\varepsilon_{t-i}
	\end{align*}
	
	Então a solução geral fica:
	
	\begin{align*}
	y_t=y_t^c+y_t^p=A_10.5^t+A_20.2^t+\frac{5}{2}+\sum\limits_{i=0}^{\infty}\Bigg[\frac{5}{3}(0.5)^i-\frac{2}{3}(0.2)^i\Bigg]\varepsilon_{t-i}
	\end{align*}

	
		
		
	\item %6b	
	\begin{align*}
			y_t &= 1 -0.3y_{t-1} + 0.1y_{t-2} + \varepsilon_t\\
			b_1&=\frac{-0.3+\sqrt{(0.3)^2+4(0.1)}}{2}=0.2\\
			b_2&=\frac{-0.3-\sqrt{(0.3)^2+4(0.1)}}{2}=-0.5\\
			\\
			y_t^c&=A_1(0.2)^t+A_2(-0.5)^t
	\end{align*}
	
	solução particular determinística
	
	\begin{align*}
	 y_t^{p,d}=b_0 \Rightarrow b_0&=1-0.3b_0+0.1b_0\\
	 y_t^{p,d}=b_0&=\frac{1}{1+0.3-0.1}=0.8333
	\end{align*}
	
	solução particular estocástica
	
	\begin{align*}
	y_t^{p,e}&=\sum\limits_{i=0}^{\infty}\alpha_i\varepsilon_{t-i}
	\end{align*}
	substituindo na equação em diferença
	\begin{align*}
	\alpha_i\sum\limits_{i=0}^{\infty}\varepsilon_{t-i}&=1-0.3\Bigg[\alpha_i\sum\limits_{i=0}^{\infty}\varepsilon_{t-1-i}\Bigg]+0.1\Bigg[\alpha_i\sum\limits_{i=0}^{\infty}\varepsilon_{t-2-i}\Bigg]+\varepsilon_t\\
	\\
	\Rightarrow \alpha_0&=1\\
	\alpha_1&=-0.3\alpha_0=-0.3\\
	\alpha_2&=-0.3\alpha_1+0.1\alpha_0\\
	&\vdots\\
	\alpha_i&=-0.3\alpha_{i-1}+0.1\alpha_{i-2}\\
	\end{align*}
	\begin{align*}
	\Rightarrow \alpha_i&=A_1(0.2)^i+A_2(-0.5)^i\\
	\alpha_0&=1 \Rightarrow A_1=1-A_2\\
	\alpha_i&=(1-A_2)(0.2)^i+A_2(-0.5)^i\\
	\alpha_1=-0.3\Rightarrow &-0.3=(1-A_2)(0.2)+A_2(-0.5)\\
	&-0.3=0.2-A_2(0.2)+A_2(-0.5)\Rightarrow A_2=\frac{5}{7} \text{ e }A_1=\frac{2}{7}\\
	\alpha_i&=\bigg(\frac{2}{7}\bigg)(0.2)^i+\bigg(\frac{5}{7}\bigg)(-0.5)^i\\
	y_t^{p,e}&=\sum\limits_{i=0}^{\infty}\Bigg[\bigg(\frac{2}{7}\bigg)(0.2)^i+\bigg(\frac{5}{7}\bigg)(-0.5)^i\Bigg]\varepsilon_{t-i}
\end{align*}
	A solução geral fica
	\begin{align*}
	y_t=y_t^c+y_t^{p,d}+y_t^{p,e}=A_1(0.2)^t+A_2(-0.5)^t+\frac{12}{10}+\sum\limits_{i=0}^{\infty}\Bigg[\bigg(\frac{2}{7}\bigg)(0.2)^i+\bigg(\frac{5}{7}\bigg)(-0.5)^i\Bigg]\varepsilon_{t-i}
	\end{align*}
\end{enumerate}
			
\end{enumerate}

